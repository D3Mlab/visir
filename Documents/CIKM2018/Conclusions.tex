%Overall, we aim for this work to bring a formal IR perspective to the important research area of AUIs while opening new research directions for the application of IR methodology.  As information
%retrieval has transformed our web experience, we believe there is
%a similar opportunity to transform the present nascent state of AUIs
%through optimization and information retrieval principles to make
%them more ubiquitous in our daily lives. 



In this paper, we have expanded on the information filtering paradigm of Belkin and Croft \citep{Belkin1992}, by  introducing a novel formulation of filtering for visual information displays (VIDs) as a ``supervised'' optimization problem given an external measure of relevance for  information elements.  We have motivated and derived the expected F1-Score as an objective criteria for filter optimization and we have experimentally demonstrated that the proposed expected F1-Score metric is a good surrogate of the ground truth F1-Score.

We introduced novel filter optimization algorithms for expected F1-Score based on two different greedy strategies (Greedy and BPS).  We have also proposed a method for optimization of expected F1-score using a MILP approach that has critically allowed us to benchmark the greedy algorithms on moderate-sized datasets. The evaluations we have performed on a variety of datasets have shown that the Greedy and BPS algorithms we have proposed are relatively efficient and provide a good approximation of the optimal MILP solution.  Somewhat surprisingly, we remark that greedy methods like BPS may even do better than exact expected F1-score optimization in high noise cases due to their more restricted search space, which prevents overfitting to this noise. 

% Important areas of future work include enhanced indexing strategies and scalability in filter selection algorithms along with the 
Important areas of future work include consideration of the role of (pseudo-)relevance and other explicit or implicit feedback methods to create a tighter and more responsive user interaction loop.
Furthermore, in combination with user studies and consideration of human factors, future work should also consider novel application-specific objectives, e.g.,  in specific visualization frameworks or based on a ranking theory of results presentation (e.g., using size or color for visual ranking emphasis).  


% For future work, we are currently integrating these algorithms into our existing user interface of graph exploration and mining. We also plan to conduct a user study in the context of natural disasters discussed on twitter to analysis our algorithms and get a real insight of their accuracy. Finally, we plan to investigate new optimization metrics based on visual rendering.