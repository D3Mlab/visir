%% LyX 2.2.1 created this file.  For more info, see http://www.lyx.org/.
%% Do not edit unless you really know what you are doing.
\documentclass[sigconf,anonymous=true,authordraft=false]{acmart}
\usepackage[latin9]{inputenc}
\usepackage{array}
\usepackage{enumitem}
\usepackage{graphicx}

\makeatletter

%%%%%%%%%%%%%%%%%%%%%%%%%%%%%% LyX specific LaTeX commands.
%% Because html converters don't know tabularnewline
\providecommand{\tabularnewline}{\\}
%% A simple dot to overcome graphicx limitations
\newcommand{\lyxdot}{.}


%%%%%%%%%%%%%%%%%%%%%%%%%%%%%% Textclass specific LaTeX commands.
\newlength{\lyxlabelwidth}      % auxiliary length 

%%%%%%%%%%%%%%%%%%%%%%%%%%%%%% User specified LaTeX commands.
%\renewcommand\footnotetextcopyrightpermission[1]{} % removes footnote with conference information in first column
%\pagestyle{plain} % removes running headers


%\makeatletter
%\renewcommand\@formatdoi[1]{\ignorespaces}
%\makeatother

\usepackage{amssymb}
\usepackage{url}
\usepackage{enumitem}
\usepackage{amstext}
\usepackage{graphicx}
\usepackage{booktabs} % For formal tables


\usepackage{subfiles}

\usepackage{subfigure}
\usepackage[ruled,vlined,linesnumbered]{algorithm2e}
\DeclareMathOperator*{\argmax}{arg\,max}

\newcommand\varlist{,\makebox[0.8em][c]{.\hfil.\hfil.},} 

% Copyright
%\setcopyright{none}
%\setcopyright{acmcopyright}
%\setcopyright{acmlicensed}
\setcopyright{rightsretained}
%\setcopyright{usgov}
%\setcopyright{usgovmixed}
%\setcopyright{cagov}
%\setcopyright{cagovmixed}


% DOI
\acmDOI{10.475/123_4}

% ISBN
\acmISBN{123-4567-24-567/08/06}

%Conference
\acmConference[WOODSTOCK'97]{ACM Woodstock conference}{July 1997}{El
  Paso, Texas USA} 
\acmYear{1997}
\copyrightyear{2016}

\acmPrice{15.00}

\fancyhead{}
\settopmatter{printacmref=false, printfolios=false}

\makeatother

\begin{document}
%\title{An Information Retrieval Perspective of Filter Selection for Adaptive User Interfaces } 
%\title{An F1-Score Optimization Based Method for Clustering Geo-Temporal Data} 
%\title{Clustering Geo-Temporal Data Using Greedy Based Algorithms By Optimizing F1-Score}
%\title{Clustering Geo-Temporal Data Using Greedy Algorithms Based on F1-Score Optimization} 
%\title{Optimizing F1-Score Using Greedy Algorithms for Clustering Geo-Temporal Data} 

% Also we take a supervised perspective vs. previously unsupervised approaches
\title{Relevance-driven F1-Score Optimization\\
for Filtering in Visual Information Displays}

\author{Mohamed Reda Bouadjenek}
%\orcid{1234-5678-9012}
\affiliation{%
  \institution{The University of Toronto}
  \streetaddress{Department of Mechanical and\\ Industrial Engineering}
  \city{Toronto} 
  \state{Ontario} 
   \postcode{M5S 3G8}
  \country{Canada}
}
\email{mrb@mie.utoronto.ca}

\author{Scott Sanner}
\affiliation{%
  \institution{The University of Toronto}
  \streetaddress{Department of Mechanical and\\ Industrial Engineering}
  \city{Toronto} 
  \state{Ontario} 
   \postcode{M5S 3G8}
  \country{Canada}
}
\email{ssanner@mie.utoronto.ca }


\author{Yihao Du}
\affiliation{%
  \institution{The University of Toronto}
 \streetaddress{Department of Mechanical and\\ Industrial Engineering}
  \city{Toronto} 
  \state{Ontario} 
   \postcode{M5S 3G8}
  \country{Canada}
 }
\email{duyihao@mie.utoronto.ca }





\newcommand{\subfour}[1]{\vspace*{3mm}{\noindent\bf #1}}  
\newcommand{\subsubfour}[1]{\vspace*{1mm}{\noindent\bf #1}} 
\newtheorem{problem}{\textbf{Problem}}

% VSI -- is it really "search"?
% VID
% Information Visualization
% AUI???
\begin{abstract}
Many applications such as real-time monitoring of social network activities or urban traffic congestion reports involve high information and cognitive load tasks.  In this context, it is common to have large-scale information visualization interfaces to concurrently display system status, alerts, and events.  However, displaying all information elements simultaneously would result in a saturated and unreadable display.  Thus, the development of automated filtering for Visual Information Displays (VIDs) are required to help focus the user's attention by restricting the display of information relevant to their current task.  
In this paper, we make key contributions to the literature on information filtering for VIDs: (1) Given that a VID will typically support many types of filtering (spatial, temporal, and content), we provide a unified framework that abstracts presentation-specific details and facilitates an optimization perspective of filtering.  
% NOTE: Should be some sort of property like monotonicity of result counts that somehow can be exploited in filter design.
(2) Within this optimization framework, we argue for the formulation of 
%Given a probabilistic relevance indicator for content importance, we provide a 
filtering as maximization of an expected F1-Score (EF1) objective subject to parameterized filter constraints.
%that can be maximized via mixed integer linear programming or efficient, approximately-optimal greedy algorithms.
% In this work, we argue that information-filtering task is central to a variety of VSIs and well-suited to an Information Retrieval (IR) perspective where we argue that a good overall objective for filter selection is F1-Score; given the absence of known Boolean relevance labels for interface elements, we instead propose optimization of a surrogate metric of expected F1-Score denoted as EF1.  
And (3) specifically for optimizing filters w.r.t.\ EF1, we contribute two efficient greedy algorithms as well as an \emph{optimal} Mixed Integer Linear Programming (MILP) formulation intended to benchmark the performance of the greedy approximations on moderately sized datasets.
We experiment with three different VID scenarios and associated datasets under a variety of settings where we vary the amount of data, noise, and relevant vs. irrelevant class imbalance in the ground truth data.  We show that EF1 is a good surrogate for optimizing F1-Score and the proposed greedy algorithms are a good approximation of the optimal MILP solution, yet much more scalable.
In summary, this work provides the first formal and unified perspective of information filtering for VIDs from an F1-Score optimization perspective along with scalable and robust algorithms that demonstrate strong performance when benchmarked against optimal results.
%we hope this work helps better connect IR and AUI research, bringing a general and formal IR perspective to this important research area while opening new research directions for the application of IR methodology to AUIs.
%the community to realize the potential of the information retrieval approach to address AUIs problems.

\noindent
{\bf Keywords:} Visual Search Interfaces; Information Filtering; Constrained Optimization.



\end{abstract}




%\keywords{Adaptive User Interfaces; Search Algorithms;  mathematical optimization for IR.}
\maketitle



\section{Introduction}
\subfile{Introduction}


\section{Framework and background}
\label{sec:Framework}
\subfile{Framework}


\section{Algorithms for Filter Selection}
\label{sec:Algorithms}
\subfile{Algorithms}

%\subsection{Optimization search}
%\subfile{Optimal}

\section{Experimental setup}
\label{sec:setup}
\subfile{Setup}


\section{Experimental evaluation}
\label{sec:Evaluation}
\subfile{Experiments}


\section{Related work}
\label{sec:RelatedWork}
\subfile{RelatedWork}

\section{Conclusions and Future work}
\label{sec:Conclusions}
\subfile{Conclusions}


\bibliographystyle{abbrv}
%\bibliographystyle{unsrt}
\bibliography{biblio}

\end{document}
