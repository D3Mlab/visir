%The ideal AUI provides users adaptive level of details to help them finish different tasks \cite{Inibhunu2016}. An information system \cite{Dumais2016} aims to re-use previously seen information on the purpose of helping user to finish the task involve history information. Additionally, another UI system \cite{Seiji2001} is implemented to improve web search based on evaluation of different Page Information Agents (PIA) by user's selection information. 

%Despite of massive researches of user-interface system development, these papers focusing on different user-interface implementation has the same title but different content with our research. We aims to explore one an algorithm under IR framework to efficiently retrieve optimal UI elements set for users' further investigation on details in the interface. 

There is a substantial body of research related to UI systems \cite{Dumais2016,Seiji2001,Inibhunu2016}. %The existing works may share the same title with our work, but have different content. 
Below, we review the major works related to aspects of this paper including information element scoring, filter selection and optimization, and evaluation of AUIs.


\subfour{Element scoring:} Different systems for managing information in visual interfaces rely on some method to score information for viewing.  In this work we assumed the scoring system was given and focused our contributions on novel filter optimization and evaluation criteria and supporting algorithms.  However, other works have focused primarily on the scoring method and could be integrated as the scoring component in our work.  For example, other approaches rely on machine learning methods for scoring, that range from decision trees \cite{Cui2008b}, k-NN \cite{Amershi2011}, rule learning \cite{Mezhoudi2013} to deep learning \cite{Harold2017}.  
Some approaches go one step further to leverage user interactions as a means of relevance feedback for refining the scoring system by interacting with the interface \cite{Harold2017,Schrier2008}  or by defining personalized re-ranking rules 
\cite{Fogarty2008}.
Finally, some recent advanced but very preliminary work has considered sequential optimization of interface adaptations based on user interactions \cite{Harold2017}.  While all of the above provided contributions orthogonal to this paper, they offer interesting potential for a tighter integration of the learning and user interaction loop in future extensions of this work.


%- For "Filter selection": What do we focus on and what do others do?  What does L2R optimize?  What did Wang and Zhu optimize?  What did the MILP optimize?  How is it technically different from our criteria?

\subfour{Filter selection:} The filter selection algorithms we presented in this paper are based on the optimization of IR evaluation metrics. Given the trivial solution of optimizing the precision (singleton) and the recall (the whole collection), we considered in this work only the optimization of F1-Score (trade-off between precision and recall). However, in the context of conventional IR models, other works have focused on the optimization of other metrics using different strategies. 
For examples, Wang and Zhu \cite{Wang2010} proposed  to use an expected score approximation to optimize Average Precision, Discounted Cumulative Gain, and Reciprocal Rank. However, the authors didn't propose a way to optimize recall and F1-Score, which are critical for our filtering problem.
Also, machine learning has been explored to optimize different metrics such as NDCG or MAP through Learning to Rank (L2R) \cite{Baeza-Yates2010}. 
However, L2R cannot be directly applied in our filter optimization problem, as the task we address is to find filter settings that optimize the Boolean metric of expected F1-Score -- not a ranking metric, although this may be part of our future work. 
As for using MILPs to optimize IR metrics, to the best of our knowledge, there is only one paper describing such a method proposed by Hansen et al. \cite{Hansen1991}. The drawback in this work is the need for ground truth relevance labels of all documents.
Finally, a lot of work has been done for visualizing large-scale information graphs using filters selected via clustering-based approaches  \cite{Ahlberg1995,Liu2014,Sankaranarayanan2009,Teitler2008,Bennamane2012,Shneiderman2013,Yifan2015,Smith2009}. However, the main difference with our work is that we have a ``supervised'' objective based on the relevance predictions, whereas clustering is inherently unsupervised.




%Finally, note that all the works described here are based on the optimization of IR metrics through textual content retrieval, whereas in our work we optimize the metrics through three  sub-filters.

%- For "Evaluation": What do we focus on and what did others do?
\subfour{Evaluation:} Usually, visual information displays (VIDs) for search or exploration-oriented tasks are evaluated based on the system effectiveness of retrieving relevant elements, e.g., alarms \cite{Amershi2011}, images \cite{Cui2008b,Fogarty2008}, textual items \cite{Tsandilas2005}, news documents \cite{Schrier2008}, 
or based on the usability of the interface \cite{Gajos2008,Gajos2010,Hartmann2008}.
Also, VIDs are usually evaluated through an off-line evaluation using specific benchmarks \cite{Amershi2011,Cui2008b,Harold2017}  or through a user study \cite{Amershi2011,Gajos2008,Hartmann2008,Fogarty2008,Tsandilas2005,Gajos2010}.
Instead, the purpose of this paper was to introduce  relevance-driven EF1 optimization algorithms to perform filter selection for VIDs; hence, our paper focuses on the evaluation of filter performance and computational time complexity for this difficult optimization problem.
%% For now just need to defend what we did.  -Scott
%with respect to a ground truth Boolean relevance criterion
% Further evaluating our general approach through user studies in specific scenarios as done by the previously cited papers can help refine the novel information retrieval methodology introduced here for specific application use cases.
%selected to focus on specific sets of elements. The filters are evaluated based on their ability to retrieve optimal sets of relevant elements.  
%Although we performed mainly an offline evaluation of our algorithms, we plan to conduct a user study on natural disasters commented on Twitter to get real insight regarding the utility of the filters suggested.


%In this paper, we are evaluating efficiency of system by investigating F1-Score of retrieval element set. The intuition is to provide user reasonable amount of possible elements to decrease their investigation time. This idea focusing on the retrieval set is employed in several papers. For example, standard accuracy of alarms in ticket system is tested in \cite{Amershi2011}. 



%Meanwhile, other papers evaluated their system in terms of user study. For example, Hartmann et. al \cite{Hartmann2008} investigate users' hit ratio of total displayed elements. Krzysztof et al studied user's utility of adaptive tool in terms of time spent in finishing the task \cite{Gajos2008}. 



%\subfour{Element retrieval in UI:} The major problem in AUI is to retrieve relevant UI elements, and this problem has been addressed as classification problem. 
%Langley \cite{Langley1997} claimed that the use of an advisory system is the key component in both informative interface and generative interface. Therefore, Cui et al. \cite{Cui2008b} created an interactive system to retrieve images, then proposed to re-rank these images based on a C4.5 decision tree model. Mezhoudi \cite{Mezhoudi2013} described a system that uses ML in an adaptive UI. The author proposed a Rule Learner (RL) that learns based on user feedback (e.g., if recommendation was accepted or cancelled). 
%Amershi et al. \cite{Amershi2011} proposed CueT, a system that allows  to group alarms into "tickets" that are to be resolved by network engineers. CueT provides recommendations on how to triage alarms; orders tickets according to relevance to incoming alarms (with a visualization showing degree of relevance).
%On the other hand, to help the user in finding relevant content, Zha et al. proposed a joint text and image query suggestion for reranking \cite{Zha2009}. The search engine not only provides a textual query suggestion but also the representative images for each suggestion. % may not be relevant.

%Generative interface which involves user's sequential interactions with the interface has also been investigated. 
%Hence, Soh et al. \cite{Harold2017} applied a deep learning method  to model user's sequential decisions via a recurrent neural-network (RNN). Also, Soh et al. \cite{Harold2015} proposed a probabilistic approaches to estimate the user's next decision under uncertainty in a system with generative interface. 


%Finally, Schrier et al. \cite{Schrier2008} proposed an interactive adaptive layout for documents in a \textquotedblleft newspaper\textquotedblright{} grid format. The system doesn't incorporate any machine learning algorithm but is rather based on rules specified. 




%The UI is a key component in a recommender system.
%It is necessary to review papers about recommendation model given its importance in informative interface. 
%One of the key component in recommendation model is similarity matrix. 
%SLIM \cite{Ning2011} proposed to learn the item similarity matrix, while 
%Soh et al. \cite{Sanner2016} proposed LREC, which leverages deep learning models esp the gated recurrent units to learn user interaction pattern to improve recommendation for personalized adaptive user interfaces. 
%Deep-learning based recommendation system is also developed. AutoSVD++ \cite{ZhangYX17} is designed by integrating deep autoencoder model in the classic SVD++ model.




%It is easy to observe that great efforts are made on improvement of element-based recommendation in previous papers. However, poor explanation ability is associated with this recommendation, since the user may not immediately grasp the mechanism of recommendation. As a result, we aimed to develop a search algorithm to provide an optimal and a self-explainable filter query. 

%\subfour{Metrics optimization in UI problem:}
%Traditionally, learning-to-rank (L2R) methods are used to optimize metrics in conventional IR settings \cite{Baeza-Yates2010}. %In this method, a structure-SVM model is trained by pair-wised data transformed from ranklist data. 
%However, L2R cannot be directly applied in our filter optimization problem, as the task we address is to find filters that fit optimal sets --- although this is may be part of our future work. 
%different filters returning different sets of elements are aimed to be compared by the model. 
%One obvious problem is unfixed size of elements sets, which leads to undetermined model complexity. 

%Wang and Zhu \cite{Wang2010} proposed another statistical method to maximize expectation of several metrics based on the joint probability of document relevance. However, authors didn't propose a way to optimize recall and F1-Score, which is critical for our filtering problem. 
%\textcolor{red}{[COMMENT: Actually, score approximation is used in this paper. So the next sentence may be taken place with its following description.]} The main reason is certainly the unknown real relevant document size. This problem is solved in our paper using score approximation.
%Meanwhile, the authors intended to retrieve documents independently, which is not the case in our filtering problem.

%As for using MILP to optimize IR metrics, to the best of our knowledge, there is only one paper describing such a method proposed by Hansen et al. \cite{Hansen1991}. The drawback in this work is also the need of real relevance label of all documents. %With our score approximation solution, the MILP method is able to be applied to optimize the metrics without any relevance label.











