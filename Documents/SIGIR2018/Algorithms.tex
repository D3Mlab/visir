This section first describes the greedy algorithms used to build filters setting for querying large data graphs in order to retrieve relevant UI elements that are of interest to the users.  Following this, we describe an optimization-based approach based on Mixed Integer Linear Programming (MILP) to benchmark the performance of the proposed greedy algorithms on moderate-sized problems.

\subsection{Greedy search}

% Should discuss greedy algorithm generically as having a metric and at each step
% a choice of k restrictions, which are each scored against the metric with the
% highest score chosen at each step.  Then each individual filter only has to
% specify what the choices are and how the choice restricts the set of emails
% selected.  What is a good succinct notation for this?

As is discussed previously, three types of sub-filters are used to select a subset of relevant information in a data visualization interface: keywords sub-filter, time sub-filter, and space (location) sub-filter.  A global filter that is generated by \emph{conjoining} these three sub-filters, is used to select the subset $RS$ of relevant UI elements. While considering a greedy top-down approach for building these filter settings, we describe in the following how we propose to build each of these sub-filters individually, and then how we propose to combine them in order to select a global filter.

\subsubsection{Keywords greedy algorithm}
The idea behind the keywords-based greedy top-down algorithm is the following: given a set of elements and an evaluation measure, the algorithm aims to select a set of keywords to form a negation query in order to exclude a subset of elements containing these keywords for the purpose of maximizing the evaluation measure.

Formally, given a keyword query representation $Q_k$, the algorithm aims to select an
optimal subset of $k$ terms $T_{k}^{*}\subset E$ (where $|T_{k}^{*}|=k$ and $E$ is the initial set of elements) to build a negation keyword query $Q_k$, i.e. $Q_k=\{\neg t_{1}^{*},\dots \neg t_{k}^{*}\}$, in order to optimize a given evaluation measure, e.g. expected precision, expected recall, expected F1-score. This is achieved by
building $T_{k}^{*}$ in a greedy manner by choosing the next optimal
term $t_{k}^{*}$ given the previous set of optimal term selections
$T_{k-1}^{*}=\{t_{1}^{*},\ldots,t_{k-1}^{*}\}$ (assuming $T_{0}^{*}=\emptyset$)
using the following selection criterion:
\begin{equation}
t_{k}^{*}=\argmax_{t_{k}\notin T_{k-1}^{*}}\hspace{-0.3mm}[EF1(E^{*} \textrm{ that satisfies } Q_k=\{\neg t_{1}^{*},\dots \neg t_{k}^{*}\})]
\end{equation}
assuming we use expected F1-score to assess  $E^{*}$, which is a subset of the initial element set $E$ that satisfy the negation keyword query $Q_k$. In order to reduce the keyword search space, we propose to use the top 100 terms ranked using mutual information. Note that the best indexing strategy to use here is obviously the inverted index data structure \cite{Zobel2006}.

\subsubsection{Time greedy algorithm}
The idea behind the time-based greedy top-down algorithm is as simple as finding a query time window range $Q_t=[t_{start},t_{end}]$, which allows to select a subset of elements $E^{*}\subseteq E$ falling in that time window, with $E^{*}$ having a higher evaluation metric value. 

Formally, given an evaluation measure (say EF1) and a list of elements $E=\{j_{t_1}\leq \dots \leq j_{t_n}\}$, where "$\leq$" specifies the timestamp order, we propose the following two time greedy algorithms:

\subfour{Naive greedy algorithm:} First, at each iteration of this algorithm, an early ranked element $j_{t_i}$ is removed, and then, the remaining set is assessed using expected F1-Score. If the remaining set has a lower expected F1-score value than the set of the previous iteration, the algorithm assigns $t_i$ to the lower time bound of the time query, i.e., $t_{start}=t_i$. 
Next, the algorithm does the same set of operations, by removing at each iteration a lastly ranked element  $j_{t_i}$, and by stopping once the removal of  $j_{t_i}$ causes a decrease in the expected F1-Score value. Then, the algorithm assigns $t_i$ to the upper time bound of the time query, i.e., $t_{end}=t_i$. 
Lastly, the algorithm returns the  time query $Q_t=[t_{start},t_{end}]$, with obviously $EF1(E^{*} \textrm{ that satisfies } Q_t=[t_{start},t_{end}]) \geq EF1(E)$.

\subfour{Dichotomy-based greedy algorithm:} Having a large dataset with a low number of positive data (e.g., 0.5\% of alerts in a security graph) will cause the previous algorithm to take a large number of iterations to terminate since it greedily adjusts filter settings in a minimal way at each step.  A way to address this problem is to use binary partitioning search, which we call Dichotomy search.  Hence, instead of removing a single element $j$ at each iteration, this algorithm operates by selecting between two distinct alternatives (dichotomies) at each iteration. 

\begin{algorithm}[t]
%\scriptsize
\caption{Dichotomy Algorithm}
\SetAlgoLined
\SetKwData{Left}{left}\SetKwData{This}{this}\SetKwData{Up}{up}
\SetKwFunction{Union}{Union}\SetKwFunction{FindCompress}{FindCompress}
\SetKwInOut{Input}{input}\SetKwInOut{Output}{output}
\Input{A set of ordered elements $E=\{j_{v_1} \dots j_{v_n}\}$ }
\Output{A timestamp $t$;}
\BlankLine
\label{alg:Dichotomy}

$v_{min}=v_1$; $v_{max}=v_n$;  $v_{mid}=\tfrac{v_1+v_n}{2}$;

\While {$v_{min}!=v_{mid}!=v_{max}$}{

\eIf{$[EF1(\{j_{v_{min}} \dots  j_{v_n}\}) \geq EF1(\{j_{v_{mid}} \dots  j_{v_n}\})]$}{
$v_{max}=v_{mid}$;
$v_{mid}=\tfrac{v_{min}+v_{mid}}{2}$;
}{
$v_{min}=v_{mid}$; 
$v_{mid}=\tfrac{v_{min}+v_{max}}{2}$;
}
}
\Return  $v_{mid}$;

\label{alg:return}
\end{algorithm} 

%Therefore, the algorithm first sets the values $t_{min}=t_1$, $t_{max}=t_n$,  and $t_{mid}=\tfrac{t_1+t_n}{2}$. Then, for each iteration, if $[EF1(\{d_{t_{min}}\leq \dots \leq d_{t_n}\}) \geq EF1(\{d_{t_{mid}}\leq \dots \leq d_{t_n}\})]$, the algorithm sets $t_{max}=t_{mid}$, $t_{mid}=\tfrac{t_{min}+t_{mid}}{2}$ and makes a new iteration, else, the algorithm sets $t_{min}=t_{mid}$,  $t_{mid}=\tfrac{t_{min}+t_{max}}{2}$  and makes a new iteration. The algorithm keeps iterating until $t_{min}=t_{mid}=t_{max}$, where it assigns $t_{mid}$ to the lower time bound of the time query, i.e., $t_{start}=t_{mid}$.


As an example of the Dichotomy approach for the time sub-filter, the algorithm first sorts $E$ in increasing order of time stamp. Then, it applies the procedure described by Algorithm \ref{alg:Dichotomy}. This procedure will return  the lower time bound of the time query, i.e., $t_{start}=t_{mid}$.
Next, the algorithm sorts $E$ in decreasing order of time stamp, and then, it applies again the procedure described by Algorithm \ref{alg:Dichotomy} to get the upper time bound of the time query, i.e., $t_{end}=t_i$. Lastly, the algorithm returns the  time query $Q_t=[t_{start},t_{end}]$, such that $EF1(E^{*} \textrm{ that satisfies } Q_t=[t_{start},t_{end}]) \geq EF1(E)$. Note that this algorithm proceeds in a total of $log(n)$ iterations in the best case, and $2\times log(n)$ iterations in the worst case.

For both the naive and time-based greedy algorithms, we use the red-black tree as the indexing data structure \cite{Guibas1978}.


\subsubsection{Position greedy algorithm} The aim of this algorithm is to return a query $Q_p=[(x_{min},y_{min}),(x_{max},y_{max})]$, which expresses the best set (assessed using one of the expected metrics described) of elements falling in the bounding box represented by the  lower and upper bound coordinates -- respectively $(x_{min},y_{min})$ and $(x_{max},y_{max})$. This two dimensional problem is similar to the previous one dimensional problem of finding the best time window. Therefore, the two greedy algorithms described above can be adapted for this problem by first applying each algorithm on the x-axis to determine $(x_{min},x_{max})$, then on the y-axis to determine $(y_{min},y_{max})$.  We omit the description of these two algorithms for lack of space, but a detailed description can be found in \cite{Bouadjenek2018}.

We use the R-tree as a data structure for indexing multi-dimensional continuous data~\cite{Guttman1984}.


\subsubsection{Multi-filter algorithm} To get a global query combining all filters, we propose a greedy algorithm, which basically at each iteration selects the best filter to apply, then checks if that filter improves the evaluation metric used. If so, the algorithm makes another iteration using the sub-set obtained in the previous iteration, otherwise, it stops. Note that here, we use  $k=1$ for the keywords greedy algorithm. The algorithm will then determine a sequence of filters to apply on the initial set, and the final query is then built by combining these filters by types.

 %such as: $\{Keyword \to Time -> Position \to Time \to Keyword\} $

For example, let's suppose the algorithm determines the following filter sequence: $\{Q_{k_1}=\{ \neg natural\} \to Q_{t_1}=[50,2030] \to Q_{p_1}=[(10,60), (50,100)] \to Q_{t_2}=[60,1230] \to Q_{k_2}=\{ \neg fictive\}  \to  Q_{p_2}=[(10,60), (30,85)] \to  Q_{t_3}=[60,800] \}$. 
The final query is built by combining these filters by types as follows: $Q_k=Q_{k_1} \cup Q_{k_2}=\{\neg natural,\neg fictive\}$, $Q_t =Q_{t_1} \cap Q_{t_2} \cap Q_{t_3}=[60,800]$, and  $ Q_p=Q_{p_1}  \cap Q_{p_2}=[(10,60),$ $(30,85)]$, which gives $Q=[Q_k=\{\neg natural,\neg fictive\}\wedge Q_t=[60,800]\wedge Q_p = [(10,60),$ $(30,85)]]$.

Finally, note that this algorithm can use the keywords greedy algorithm with the time and position naive greedy algorithms to which we refer as Greedy Algorithm, or the keywords greedy algorithm with the time and position dichotomy-based algorithms to which we refer as Dichotomy Algorithm.



\subsection{Optimal Solutions for Benchmarking}

Next we propose an optimization-based algorithm to maximize EF1 and provide a benchmark for evaluation of the previous Greedy algorithms.  Given the trivial solution of optimizing the expected precision (singleton) and expected recall (the whole collection), we consider in the following only the optimization of EF1.  

\subsubsection{Original intuitive formulation} \hfill \\
Let's first reformulate the objective EF1 to prepare further optimization step. Note that the global sum of scores of all information elements is a constant $C$ given the current system.
\begin{equation}
\begin{aligned}
    \emph{$EF1$} &= \dfrac{2\times \sum_{j=1}^m S(j)I(j)}{\sum_{j=1}^m I(j) + \sum_{j=1}^m S(j)} = \dfrac{2 \times \sum_{j=1}^m S(j)I(j)}{\sum_{j=1}^m I(j) + C}
\end{aligned}
\end{equation}

In order to obtain the optimal element set, we intend to directly optimize the EF1 metric in terms of decision indication variables $\emph{I\textsubscript{filter}} \in \left[0, 1\right]$ for filter setting as follows:
% Don't use I(i)... confusing!  
%\begin{align}
%\max_{\textit{filter vars}} \;\;
%& \dfrac{\sum_{j=1}^m S(i)I(j)}{\sum_{j=1}^m I(j) + C} \nonumber \\
%& \textrm{subject to constraints between {\it filter vars} and $I(j)$}   
%\end{align}
\begin{equation}
\begin{aligned}
& \underset{I_{\mathit{filter}}(j)}{\text{maximize}}
& & \dfrac{\sum_{j=1}^m S(j)I(j)}{\sum_{j=1}^m I(j) + C} \\
& s.t
& & I(j) = \bigwedge I_{\mathit{filter}}(j) \\
\end{aligned}
\end{equation}

Note that our goal is to optimize the element set in terms of \emph{filters settings}. This means elements in the interface sharing the same property needs to be simultaneously added to the selected set. This is a unique property of our filter-based UI problem, which is different from the independent retrieval of each document in standard IR system.

\subsubsection{Transformation to a MILP} \hfill \\
In order to transform the above fractional optimization problem into a Mixed-integer Linear Programming (MILP) form (for which we have optimal solvers), we use the Charnes-Cooper method \cite{Charnes1962} and Glover linearization method \cite{Glover1975} with big-M constraints, where auxiliary variables \emph{w(j)} and \emph{u} must be  introduced\footnote{https://optimization.mccormick.northwestern.edu/index.php/Mixed-integer\_linear\_fractional\_programming\_(MILFP)}. Here, $w(j)$ is defined as $w(j)=I(j)\times u$ with $u$ defined as follows:
\begin{equation}
u = \dfrac{1}{\sum_{j=1}^m I(j) + C}
\end{equation}

Then, the EF1 optimization problem is able to be transformed into the following MILP problem:
\begin{equation}
\begin{aligned}
& \underset{w,u}{\text{maximize}}
& & \sum_{j=1}^m S(j)w(j) \\
& s.t
& & \sum_{j=1}^m w(j) + uC = 1 \\
& & & w(j) \leqslant u, \quad w(j) \leqslant M\times I(j)  \\
& & & w(j) \geqslant u - M\times [1-I(j)] \\
& & & u > 0,  \quad I(j) \in \{0, 1\}, \quad w(j) \geqslant 0
\end{aligned}
\end{equation}

\subsubsection{Constraints} \hfill \\
As our goal is to select elements through three sub-filters, we add three constraints to the above optimization.
\begin{enumerate}
\item Time: a two-element tuple ($t_{start}$, $t_{end}$) indicating respectively the start and the end of the time window.

\begin{equation}
\begin{aligned}
  I_{\mathit{time}}(j) &=
   \begin{cases}
     1, & \text{if $(t_{start} \leqslant t(j)) \land (t(j) \leqslant t_{end})$}  \\
     0, & \text{otherwise}
  \end{cases}
\end{aligned}
\end{equation}

\item Position: a four-element tuple ($x_{min}$, $y_{min}$, $x_{max}$, $y_{max}$) to create a bounding box filter in visualization interface.

\begin{equation}
\begin{aligned}
I_{\mathit{pos}}(j) & =\begin{cases}
1, & \text{if \ensuremath{(x_{min}\leqslant x(j))\land(x(j)\leqslant x_{max})\land}}\\
 & (y_{min}\leqslant y(j))\land(y(j)\leqslant y_{max})\\
0, & \text{otherwise}
\end{cases}
\end{aligned}
\end{equation}

\item Keyword: a boolean vector of terms $t^*_k$ with size $m$ - the size of the dictionary of the global collection.

\begin{equation}
  I_{\mathit{term}}(j) = \bigwedge_{t^*_k \in j} t^*_k \qquad \textnormal{for s = 1, 2, $\cdots$, m}
\end{equation}
All terms with $I_{\mathit{term}}=0$ are included in the negation query.


\item Multiple Filter: if an element $j$ is simultaneously selected via the three sub-filters, all filter constraints have to be satisfied in this element. In others words, an AND operator is required between all the sub-filter constraints.

%{\bf TODO: mention how to apply to all filters... need to say an email j is selected if \emph{all} filters say it is selected, so an AND constraint.  \textcolor{red}{[PLEASE COMPLETE YIHAO]}.
%} 

\begin{equation}
  I(j) = I_{\mathit{time}}(j) \land I_{\mathit{pos}}(j) \land I_{\mathit{keyword}}(j)
\end{equation}

\end{enumerate}

\subsection{Multiple Filter Selection Wrapper}

In practice a single filter chosen by the previously described algorithms will narrow the user in on a single \textquotedblleft event\textquotedblright{}. However, there will likely be multiple anomalous events and so the user should have a choice of multiple filters.  In this work, we provide an initial greedy approach for providing a ranked list of multiple filters (that is a wrapper approach working with any of the previously defined filtering algorithms); we leave it to future work to develop improved filtering and ranking methods for multiple filters.

The algorithm itself is quite simple.  After the first filter is produced, all selected elements by the filter have their scores zeroed out.  The filtering algorithm is then run again, where it will inherently focus on a different content set.  This procedure is repeated until the desired number of filters is reached, or a metric / coverage score for high scoring content is reached.  The user should then be able to choose among the multiple filters. 	










